Title: Bogoliubov-Valatin transformation Date: 2019-11-28 11:20
Category: Physics Tags: Quantum physics, Superconductivity Slug:
homework Author: Rémy Torro Summary: Derivation of thermodynamic
quantities in the Bogoliubov Valatin formalism.

 In 1947, Bogoliubov introduced a new linear transformation to
diagonalize the quadratic Hamiltonian in superconductivity. The methods
was extended later by Valatin and Bogoliubov to the Fermi case in the
theory of superconductivity ({[}@xiao\_theory\_2009{]}).

\textbf{1) Write the full Hamiltonian in term of \(\gamma\),
\(\gamma^\dagger\).}

The part of the interaction Hamiltonian which contributes to
superconductivity, \emph{i.e.} the attractive part leading to the
formation of pairs with opposite momenta and opposite spins, can be
written as:

\[ \hat{H}_{R}' = \hat{H}_{0}' + \hat{H}_{\textrm{IR}}\]

\[ \hat{H}_0' = \sum_{k\sigma} \xi_k c_{k\sigma}^\dagger c_{k\sigma}  = \sum_k \xi_k [c_{k\uparrow}^\dagger c_{k\uparrow} + c_{-k\downarrow}^\dagger c_{-k\downarrow}]\]

\[ \hat{H}_{\textrm{IR}} = \sum_{k,k'} V_{kk'} c_{k'\uparrow}^\dagger c_{-k'\downarrow}^\dagger c_{-k\downarrow} c_{k\uparrow}\]

We will use the Bogoliubov-Valatin transformation which can be written
as follow:

\[\begin{pmatrix}c_{k\uparrow}^\dagger \\ c_{-k\downarrow}\end{pmatrix} = \begin{pmatrix}u_k & v_k \\ -v_k & u_k \end{pmatrix}\begin{pmatrix}\gamma_{k\uparrow}^\dagger \\ \gamma_{-k\downarrow}\end{pmatrix}\]

In the following, we will assume that \(u_k\) and \(v_k\) are real
coefficients. The reduced Hamiltonian of the system can be rewritten in
terms of the Bogoliubov quasiparticle operators. In doing so, we
generate two types of terms: those containing products of an equal
number of creation \(\gamma^\dagger_{k\sigma}\) and destruction
\(\gamma_{k\sigma}\) operators and those containing an unequal number.
When taking thermally averaged quantum-expectation values of such
products, only the first forms yield nonzero averages.

\[ \hat{H}_0' - (\textrm{terms with unequal numbers of $\gamma^\dagger_{k\sigma}$ and $\gamma_{k\sigma}$})  = \sum_k \xi_k \left[(u_k \gamma_{k\uparrow}^\dagger+v_k\gamma_{-k\downarrow})(u_k \gamma_{k\uparrow}+v_k\gamma_{-k\downarrow}^\dagger) + (u_k\gamma_{-k\downarrow}^\dagger - v_k \gamma_{k\uparrow})(u_k \gamma_{-k\downarrow}-v_k\gamma_{k\uparrow}^\dagger) \right] \\ = \sum_k \xi_k \left[u_k^2 \gamma_{k\uparrow}^\dagger \gamma_{k\uparrow} + v_k^2 \gamma_{-k\downarrow}\gamma_{-k\downarrow}^\dagger + u_k^2\gamma_{-k\downarrow}^\dagger\gamma_{-k\downarrow} + v_k^2\gamma_{k\uparrow}\gamma_{k\uparrow}^\dagger \right] \\ = \sum_k \xi_k \left[u_k^2 (\gamma_{k\uparrow}^\dagger \gamma_{k\uparrow}+\gamma_{-k\downarrow}^\dagger\gamma_{-k\downarrow}) + v_k^2(1 - \gamma_{-k\downarrow}^\dagger\gamma_{-k\downarrow}) + v_k^2(1 - \gamma_{k\uparrow}^\dagger\gamma_{k\uparrow})\right] \\ = \sum_k \xi_k \left[2 v_k^2 + (u_k^2 - v_k^2)(\gamma_{k\uparrow}^\dagger\gamma_{k\uparrow}+\gamma_{-k\downarrow}^\dagger\gamma_{-k\downarrow})\right]\]

\[ \hat{H}_{\textrm{IR}} - (\textrm{terms with unequal numbers of $\gamma^\dagger_{k\sigma}$ and $\gamma_{k\sigma}$}) = \sum_{k,k'} V_{kk'} (u_{k'}\gamma_{k'\uparrow}^\dagger + v_{k'}\gamma_{-k'\downarrow})(u_{k'}\gamma^\dagger_{-k'\downarrow}-v_{k'}\gamma_{k'\uparrow})(u_k\gamma_{-k\downarrow}-v_k\gamma_{k\uparrow}^\dagger)(u_k\gamma_{k\uparrow}+v_k\gamma_{-k\downarrow}^\dagger) \\ = \sum_{k,k'} V_{kk'} u_{k'}v_{k'}u_k v_k \left[-\gamma_{k'\uparrow}^\dagger \gamma_{k'\uparrow}\gamma_{-k\downarrow}\gamma_{-k\downarrow}^\dagger + \gamma_{k'\uparrow}^\dagger\gamma_{k'\uparrow}\gamma_{k\uparrow}^\dagger\gamma_{k\uparrow} + \gamma_{-k'\downarrow}\gamma_{-k'\downarrow}^\dagger \gamma_{-k\downarrow}\gamma_{-k\downarrow}^\dagger - \gamma_{-k'\downarrow}\gamma_{-k'\downarrow}^\dagger \gamma_{k\uparrow}^\dagger \gamma_{k\uparrow} \right]
\\ = \sum_{k,k'}V_{kk'} u_{k'}v_{k'}u_k v_k \left[-\gamma_{k'\uparrow}^\dagger\gamma_{k'\uparrow}(1-\gamma_{-k\downarrow}^\dagger\gamma_{-k\downarrow}) + \gamma_{k'\uparrow}^\dagger \gamma_{k'\uparrow} \gamma_{k\uparrow}^\dagger \gamma_{k\uparrow} + (1-\gamma_{-k'\downarrow}^\dagger\gamma_{-k'\downarrow})(1 - \gamma_{-k\downarrow}^\dagger \gamma_{-k\downarrow}) - (1-\gamma_{-k'\downarrow}^\dagger\gamma_{-k'\downarrow})\gamma_{k\uparrow}^\dagger\gamma_{k\uparrow}  \right] \\ = \sum_{k,k'}V_{kk'} u_k v_k u_{k'}v_{k'} (1 - \gamma_{k\uparrow}^\dagger \gamma_{k\uparrow} - \gamma_{-k\downarrow}^\dagger\gamma_{-k\downarrow})(1 - \gamma_{k'\uparrow}^\dagger \gamma_{k'\uparrow}-\gamma_{-k'\downarrow}^\dagger \gamma_{-k'\downarrow})\]

\[ \hat{H}'_R = \hat{H}'_0 + \hat{H}_{\textrm{IR}} \\ = \sum_k \xi_k [2 v_k^2 + (u_k^2 - v_k^2)(\gamma^\dagger_{k\uparrow}\gamma_{k\uparrow}+\gamma^\dagger_{-k\downarrow}\gamma_{-k\downarrow})] + \sum_{k,k'} V_{k k'} u_k v_k u_{k'}v_{k'}(1 - \gamma^\dagger_{k\uparrow}\gamma_{k\uparrow}-\gamma^\dagger_{-k\downarrow}\gamma_{-k\downarrow})(1 - \gamma^\dagger_{k'\uparrow}\gamma_{k'\uparrow}-\gamma^\dagger_{-k'\downarrow}\gamma_{-k'\downarrow})+ \textrm{terms with unequal numbers of $\gamma^\dagger_{k\sigma}$ and $\gamma_{k\sigma}$}\]

\textbf{2) Use Wick theorem to compute thermal averages.}

Firstly, let's compute the thermal average of \(\hat{H}_0'\):

\[\langle \hat{H}_0' \rangle = \sum_k \xi_k [2 v_k^2 + (u_k^2 - v_k^2)(\langle \gamma_{k\uparrow}^\dagger \gamma_{k\uparrow}\rangle + \langle \gamma_{-k\downarrow}^\dagger \gamma_{-k'\downarrow}\rangle)]\]

Since the fermion-quasiparticle excitations do not interact (in our
mean-field-like theory), the averages are given by
\(\langle \gamma^\dagger_{k\sigma}\gamma_{k\sigma} \rangle = f_{k\sigma}\).
Let's assume that \(2 f_k = f_{k\uparrow}+f_{-k\downarrow}\). As a
result:

\[ \langle \hat{H}_0' \rangle = \sum_k \xi_k [2 v_k^2 + (u_k^2 - v_k^2)2 f_k] = 2 \sum_k \xi_k [v_k^2 + (u_k^2 - v_k^2)f_k]\]

Now we have to deal with the thermal average of
\(\hat{H}_{\textrm{IR}}\). The parentheses in equation \ldots{} can be
developped as:
\(I = 1 - \gamma_{k'\uparrow}^\dagger \gamma_{k'\uparrow} - \gamma_{-k'\downarrow}^\dagger \gamma_{-k'\downarrow} - \gamma_{k\uparrow}^\dagger \gamma_{k\uparrow} + \gamma_{k\uparrow}^\dagger\gamma_{k\uparrow}\gamma_{k'\uparrow}^\dagger\gamma_{k'\uparrow} + \gamma_{k\uparrow}^\dagger\gamma_{k\uparrow}\gamma_{-k'\downarrow}^\dagger\gamma_{-k'\downarrow} - \gamma_{-k\downarrow}^\dagger\gamma_{-k\downarrow}+\gamma_{-k\downarrow}^\dagger\gamma_{-k\downarrow}\gamma_{k'\uparrow}^\dagger\gamma_{k'\uparrow}+\gamma_{-k\downarrow}^\dagger\gamma_{-k\downarrow}\gamma_{-k'\downarrow}^\dagger\gamma_{-k'\downarrow}\)

The thermal average of couples of fermionic operators is computed using
\(\langle \gamma^\dagger_{k\sigma}\gamma_{k\sigma} \rangle = f_{k\sigma}\).
For quartets of fermionic operators, we can apply Wick's theorem. Let's
compute the average for the first of these quartets:

\[\langle \gamma_{k\uparrow}^\dagger \gamma_{k\uparrow}\gamma_{k'\uparrow}^\dagger\gamma_{k'\uparrow} \rangle = \langle  \gamma_{k\uparrow}^\dagger \gamma_{k\uparrow} \rangle\langle \gamma_{k'\uparrow}^\dagger \gamma_{k'\uparrow} \rangle 
- \langle \gamma_{k\uparrow}^\dagger\gamma_{k'\uparrow}^\dagger \rangle \langle \gamma_{k\uparrow}\gamma_{k'\uparrow}\rangle 
+ \langle \gamma_{k\uparrow}^\dagger \gamma_{k'\uparrow}\rangle \langle \gamma_{k\uparrow}\gamma_{k'\uparrow}^\dagger \rangle
\\ = f_{k\uparrow}f_{k'\uparrow} - 0 + 0 \]

We can proceed in the same fashion for each term and get:

\[\langle I \rangle = 1 -f_{k'\uparrow} -f_{-k'\downarrow} - f_{k\uparrow} - f_{-k\downarrow} + f_{k\uparrow}f_{k'\uparrow} + f_{k\uparrow}f_{-k'\downarrow}+f_{-k\downarrow}f_{k'\uparrow}+f_{-k\downarrow}f_{-k'\downarrow}\]

Assuming that
\(f_k f_{k'} = \frac{1}{4}(f_{k\uparrow}+f_{-k\downarrow})(f_{k'\uparrow}f_{-k'\downarrow})\),
we recognize that we have
\(\langle I \rangle = 1 - 2f_{k'}-2f_k + 4f_kf_{k'}\).

Eventually:

\[\langle \hat{H}_{\textrm{IR}}' \rangle = \sum_{k,k'} V_{kk'} u_k v_k u_{k'}v_{k'}(1-2f_k)(1-2f_{k'})\]

Gathering all the terms, we can write:

\[\langle \hat{H}_R' \rangle = 2\sum_k \xi_k[v_k^2 + (u_k^2 - v_k^2)f_k] + \sum_{k,k'} V_{kk'}u_k v_k u_{k'}v_{k'}(1-2f_k)(1-2f_{k'})\]

\textbf{3) Use microcanonical argument to find the entropy.}

There can be no more than one particle in each quantum state. The number
of possible ways of distributing \(N_j\) identical particles among
\(G_j\) states with no more than one particle in each is the number of
ways of selecting \(N_j\) of the \(G_j\) states
({[}@landau\_statistical\_2013{]}):

\[ \Delta \Gamma_j = \frac{G_j !}{N_j!(G_j - N_j)!}\]

If we take the logarithm of this quantity and apply Stirling's
approximation, we can write the following ``microcanonical'' entropy:

\[S = \ln \prod_j \Delta \Gamma_j \\ = \sum_j \ln \frac{G_j !}{N_j!(G_j - N_j)!} = \sum_j \left\{G_j \ln G_j - N_j \ln N_j -(G_j - N_j)\ln(G_j - N_j)-G_j + N_j + (G_j - N_j)  \right\} \\ = \sum_j \left\{G_j \ln G_j - N_j \ln N_j -(G_j - N_j)\ln(G_j - N_j) \right\}\]

Using the mean occupation numbers of the quantum states
\(n_j = N_j / G_j\), we can rewrite the above expression as:

\[S = \sum_j \left\{G_j \ln G_j - G_j n_j \ln(G_j n_j) -(G_j - G_j n_j)\ln(G_j - G_j n_j) \right\} \\ = \sum_j G_j \left\{\ln G_j - n_j \ln(G_j n_j) -(1 - n_j)\ln[G_j(1-n_j)] \right\} \\ = \sum_j G_j \left\{\ln G_j - n_j \ln G_j - n_j \ln n_j -(1 - n_j)\ln G_j - (1 -n_j)\ln(1 - n_j) \right\} \\ = -\sum_j G_j \left\{n_j \ln n_j + (1 -n_j)\ln(1 - n_j) \right\} \]

The distribution \(n_j\) that maximises this entropy is the usual
equilibrium Fermi-Dirac distribution \(n_j = f_j\). For electrons, we
can set the degeneracy \(G_j = 2\). If we introduce \(k_B\) to get the
right units;

\[S = -2 k_B \sum_k \left[f_k \ln f_k + (1 -f_k)\ln(1 - f_k) \right] \]

\textbf{4) Find the gap self-consistent equation.}

We can introduce \(u_k = \cos \theta_k\) and \(v_k = \sin \theta_k\) and
minimize \(F\) with respect to \(\theta_k\)
(\(\partial F / \partial \theta_k = 0\)). Since the \(-TS\) term in
\(F\) does not depend on \(\theta_k\) we can immediately compute the
following derivative:

\[\frac{\partial \langle H_R' \rangle }{\partial \theta_k} = 2\xi_k [2\cos \theta_k \sin \theta_k - 4 \cos \theta_k \sin \theta_k f_k] + 2 \underbrace{\sum_{k'} V_{kk'}(1-2 f_{k'})\cos \theta_{k'}\sin \theta_{k'}}_{-\Delta_k}(\cos^2 \theta_k - \sin^2 \theta_k)(1-2 f_k)  \\ = 4 \xi_k u_k v_k (1-2f_k) - 2 \Delta_k (u_k^2 - v_k^2)(1-2f_k) = 0\]

where we introduced the gap function
\(\Delta_k = -\sum_{k'} V_{kk'}u_{k'}v_{k'}(1 - 2f_{k'})\). Using the
property \(u_k^2 + v_k^2 = 1\), we can write \(v_k^2 = 1-u_k^2\) and
substitute:

\[ 2\xi_k u \sqrt{1 - u^2} = \Delta_k (2 u^2 -1)\]

\[ 4 \xi_k^2 u^2 (1 - u^2) = \Delta_k^2 [4u^4 - 4u^2 +1]\]

\[ u^4 - u^2 + \frac{\Delta_k^2}{4(\xi_k^2 + \Delta_k^2)} = 0\]

We can compute the discriminant \(\delta\):

\[ \delta = 1 - \frac{\Delta_k^2}{\xi_k^2 + \Delta_k^2} = \frac{\xi_k^2}{\xi_k^2 + \Delta_k^2}\]

This gives us the following solutions for \(u^2\):

\[ u^2 = \frac{1}{2}\left(1 \pm \sqrt{\frac{\xi_k^2}{\xi_k^2 + \Delta_k^2}}\right) = \frac{1}{2}\left(1 \pm \frac{\xi_k}{\varepsilon_k}\right)\]

where we introduced the energy
\(\varepsilon_k = \sqrt{\xi_k^2 + \Delta_k^2}\), which is the energy
required to add an electron to the system in a state \(k\). If we
substitute \(u_k^2 = 1 - v_k^2\) we find:

\[ v_k^2 = \frac{1}{2}\left(1 \pm \frac{\xi_k}{\varepsilon_k}\right)\]

We can select the upper sign for \(u_k\) and the lower sign for \(v_k\)
in order to satisfy \(u_k^2 + v_k^2 = 1\).

Using \(f_{k'} = 1/(e^{\varepsilon_{k'}/k_B T}+1)\), we can rewrite
\((1-2f_{k'})\) in \(\Delta_k\) as:

\[ (1-2 f_{k'}) = 1 - \frac{2}{e^{\varepsilon_{k'}/k_BT}+1} = \frac{e^{\varepsilon_{k'}/k_B T}+1 - 2}{e^{\varepsilon_{k'}/k_B T}+ 1} = \frac{e^{\varepsilon_{k'}/k_B T}-1}{e^{\varepsilon_{k'}/k_B T}+1} = \tanh \left(\frac{\varepsilon_{k'}}{2 k_B T} \right)\]

And as a result:

\[ \Delta_k = \sum_{k'} V_{kk'} u_{k'}v_{k'}\tanh \left(\frac{\varepsilon_{k'}}{2 k_B T} \right)\]

Using the expressions for \(u_k\) and \(v_k\) we may rewrite the gap
function as:

\[ \Delta_k = -\sum_{k'}V_{kk'} \sqrt{\frac{1}{2}\left(1+\frac{\xi_{k'}}{\varepsilon_{k'}}\right)\frac{1}{2}\left(1-\frac{\xi_{k'}}{\varepsilon_{k'}}\right)}(1 - 2 f_{k'}) \\ = -\sum_{k'}V_{kk'} \sqrt{\frac{1}{4}\left(1-\frac{\xi_{k'}^2}{\xi_{k'}^2 + \Delta_{k'}^2}\right)}(1 - 2 f_{k'}) \\ = -\sum_{k'}V_{kk'} \frac{1}{2}\frac{\Delta_{k'}}{\sqrt{\xi_k^2 + \Delta_k^2}}(1 - 2 f_{k'}) \\ = - \sum_{k'} V_{kk'}\frac{(1 - 2 f_{k'})}{2\varepsilon_{k'}}\Delta_{k'} \]

We have derived the gap self-consistent equation.

\textbf{5) Find the critical temperature.}

A simple application would be to use the Cooper model potential.

\[V_{kk'}=\left\{
                \begin{array}{ll}
                  -V \quad \textrm{for} \ |\xi_k| \ \textrm{and} \ |\xi_{k'}|\leq \hbar \omega_{D} \\
                  0 \quad \textrm{for} \ |\xi_k| \ \textrm{or} \ |\xi_{k'}| > \hbar \omega_D 
                \end{array}
              \right.
\]

The function \(\Delta_k\) will have the form: \(\Delta_k = \Delta\) for
\(|\xi_k| < \hbar \omega_D\) and \(\Delta_k = 0\), for
\(|\xi_k| > \hbar \omega_D\). Thus we can integrate equation \ldots{}
between \(-\hbar \omega_D\) and \(\hbar \omega_D\) and substitute
\(\Delta_k = \Delta\):

\[\Delta \sim \mathcal{N}(0) V \Delta \int_{-\hbar \omega_D}^{\hbar \omega_D} d\xi \frac{1 - 2 f[\varepsilon]}{2\varepsilon}\]

\[ 1 = \mathcal{N}(0) V \int_{-\hbar \omega_D }^{\hbar \omega_D} d\xi \frac{1 - 2 f[\sqrt{\xi^2 + \Delta^2(T)}]}{2\sqrt{\xi^2 + \Delta^2(T)}}\]

The critical temperature \(T_c\) corresponds to \(\Delta(T_c) = 0\):

\[ 1 = \mathcal{N}(0) V \int_{-\hbar \omega_D }^{\hbar \omega_D} d\xi \frac{1 - 2 f[\xi,T_c]}{2\xi} = \mathcal{N}(0) V \int_{-\hbar\omega_D}^{\hbar \omega_D} \frac{d\xi}{2\xi} \tanh\left(\frac{\xi}{2k_B T_c}\right) = \mathcal{N}(0) V \int_0^{\hbar \omega_D} \frac{d\xi}{\xi} \tanh\left(\frac{\xi}{2k_B T_c}\right) \]

where I split the integral because the integrand \(\tanh(\xi)/\xi\) is
even. Since:

\[ \lim_{\xi \rightarrow \infty} \tanh\left(\frac{\xi}{2k_B T}\right) = 1\]

and the integral has the asymptotic form
\(\ln (\hbar \omega_D/k_B T_c) + C\); a numerical computation yields
\(C = \ln 1.14\).

\textbf{6) Compute the Free energy.}

We can express the entropy in a more convenient form as:

\[- TS = \sum_k \left[\varepsilon_k (1 - 2 f_k) - 2 k_B T \ln \left(2 \cosh \frac{\varepsilon_k}{2 k_B T}\right) \right]\]

\textbf{Proof}: let's develop this expression using
\(\cosh x = \frac{e^x + e^{-x}}{2}\):

\[ - TS = \sum_k \varepsilon_k (1 - 2 f_k) - 2 k_B T \ln [e^{\varepsilon_k/2 k_B T}+e^{-\varepsilon_k/2 k_B T}]\]

The logarithm can be rewritten as:
\(\ln [e^{-\varepsilon_k/2 k_B T}(1+e^{\varepsilon_k/k_B T})] = -\frac{\varepsilon_k}{2 k_B T} - \ln f_k\).
As a result:

\[ -TS = \sum_k \left[\varepsilon_k (1 - 2 f_k) + \varepsilon_k + 2 k_B T \ln f_k \right]\]

We will now try to derive this expression starting from equation
\ldots{} that we derived in question 3.

\[ - TS = 2 k_B T \sum_k \left\{f_k \ln f_k + (1 - f_k)\ln(1 - f_k) \right\} \\ = 2k_B T \sum_k \left\{\frac{\varepsilon_k}{k_B T} + \ln f_k - \frac{\varepsilon_k}{k_B T}f_k \right\} \\ = \sum_k \left[2 \varepsilon_k + 2 k_B T \ln f_k - 2 \varepsilon_k f_k \right] \\ = \sum_k \left[\varepsilon_k (1 - 2 f_k) + \varepsilon_k + 2 k_B T \ln f_k \right]\]

Both expressions are thus equivalent. Now, we would like to compute
\(\langle \hat{H}_{\textrm{IR}} \rangle\). We already computed it in
question 1. We want to show that it can be written as:

\[ \langle \hat{H}_{\textrm{IR}} \rangle = - \sum_k \frac{\Delta_k^2}{2\varepsilon_k}(1 - 2 f_k)\]

Starting from equation \ldots{} and substituting the expressions for
\(u_k\) and \(v_k\) derived in \ldots{} :

\[ \langle \hat{H}_{\textrm{IR}} \rangle = \sum_{k,k'} V_{k,k'} u_k v_k u_{k'}v_{k'}(1 - 2 f_k)(1-2 f_{k'}) \\ = -\sum_k \Delta_k u_k v_k (1-2 f_k) \\ = -\sum_{k} \Delta_k (1 - 2 f_k)\frac{1}{2}\underbrace{\sqrt{1 - \frac{\xi_k^2}{\varepsilon_k^2}}}_{\Delta_k / \varepsilon_k} \\ = -\sum_k \frac{\Delta_k^2}{2\varepsilon_k}(1 - 2 f_k) \\ = -\frac{\Delta^2(T)}{V}\]

where in the last line we introduced a new notation following Ketterson.
In the same way, we can compute \(\langle \hat{H}_0'\rangle\):

\[ \langle \hat{H}_0' \rangle = 2 \sum_k \xi_k \left[v_k^2 + (u_k^2 - v_k^2)f_k \right] = 2\sum_k \left[\xi_k v_k^2 + \frac{\xi_k^2}{\varepsilon_k}f_k \right] = 2 \sum_k \left[\xi_k v_k^2 + \left(\varepsilon_k - \frac{\Delta_k^2}{\varepsilon_k}\right) f_k\right]\]

Indeed:
\(\varepsilon_k - \frac{\Delta_k^2}{\varepsilon_k} = \frac{\varepsilon_k^2 - \Delta_k^2}{\varepsilon_k} = \frac{\xi_k^2}{\varepsilon_k}\).The
energy
\(E' = \langle \hat{H}_R' \rangle = \langle \hat{H}_0' \rangle + \langle \hat{H}_{\textrm{IR}} \rangle\)
can be written as:

\[E' = 2 \sum_k \left[\xi_k v_k^2 + f_k \varepsilon_k - \frac{\Delta_k^2}{\varepsilon_k} f_k \right] + \frac{\Delta^2(T)}{V} \\ = \sum_k \left[2\xi_k v_k^2 + 2f_k \varepsilon_k - 2 \frac{\Delta_k^2}{\varepsilon_k} f_k -\frac{\Delta_k^2}{\varepsilon_k}(1-2 f_k)\right] - \frac{\Delta^2(T)}{V} \\ = \sum_k \left[2\xi_k v_k^2 + 2f_k \varepsilon_k - \frac{\Delta_k^2}{\varepsilon_k}\right] - \frac{\Delta^2(T)}{V}\]

Eventually, one can write the free energy as:

\[ F = E' - TS \\ = \sum_k \left[2\xi_k v_k^2 + 2f_k \varepsilon_k - \frac{\Delta_k^2}{\varepsilon_k} \right] - \frac{\Delta^2(T)}{V} + \sum_k \left[\varepsilon_k (1 - 2 f_k) - 2 k_B T \ln \left(2 \cosh \frac{\varepsilon_k}{2 k_B T}\right) \right] \\ = \sum_k \left[2\xi_k v_k^2 - \frac{\Delta_k^2}{\varepsilon_k} + \varepsilon_k - 2 k_B T \ln \left(2 \cosh \frac{\varepsilon_k}{2 k_B T}\right)\right]-\frac{\Delta^2(T)}{V}\]

This is the Free energy of the system.

\textbf{7) Find the low temperature behavior of specific heat.}

The specific heat is defined as:

\[ C = T \frac{d S}{d T}\]

Using expression \ldots{} for the entropy, and since only \(f_k\)
depends on \(T\), we can perform the derivative:

\[ \frac{dS}{dT} = -2 k_B \sum_k \left[\frac{\partial f_k}{\partial T} \ln f_k + \frac{\partial f_k}{\partial T} - \frac{\partial f_k}{\partial T} \ln(1 - f_k) - \frac{\partial f_k}{\partial T} \right] = - 2 k_B \sum_k \left[\ln f_k - \ln(1 - f_k) \right]\frac{\partial f_k}{\partial T}\]

As a result:

\[ C = T \frac{dS}{dT} = -2 k_B T \sum_k \left[\ln f_k - \ln(1-f_k) \right]\frac{\partial f_k}{\partial T}\]

Notice that
\(\ln f_k - \ln(1-f_k) = -\ln(1+e^{\varepsilon_k/k_B T}) - \ln \left(\frac{e^{\varepsilon_k/k_B T}}{e^{\varepsilon_k/k_B T}+1} \right) = -\frac{\varepsilon_k}{k_B T}\).
Thus, we can write:

\[C = 2 k_B T \sum_k \frac{\varepsilon_k}{k_B T} \frac{d}{dT}\left(\frac{1}{e^{\varepsilon_k/k_B T}+1}\right) \]

Now, assuming that \(\varepsilon_k = \varepsilon_k(T)\) \emph{via}
\(\Delta(T)\), we can develop \(\partial f_k / \partial T\):

\[ \frac{\partial f_k}{\partial T} = \frac{d}{dT} \left(\frac{1}{e^{\varepsilon_k/k_B T}+1}\right) = \frac{-\frac{d}{dT}[e^{\varepsilon(T)/k_B T}]}{(e^{\varepsilon_k(T)/k_B T}+1)^2}
= \frac{e^{\varepsilon_k/k_BT}\left(\frac{\varepsilon_k}{k_B T^2} - \frac{1}{k_B T}\frac{\partial \varepsilon_k(T)}{\partial T}\right)}{(e^{\varepsilon_k(T)/k_B T}+1)^2} \]

Since \(\varepsilon_k = \sqrt{\Delta^2 + \xi_k^2}\), we can easily
deduce that
\(\partial \varepsilon_k / \partial T = \frac{\Delta(T)}{\varepsilon_k}\frac{d\Delta}{dT}\).
Thus we can write:

\[\frac{\partial f_k}{\partial T} = \frac{e^{\varepsilon_k/k_BT}\left(\frac{\varepsilon_k}{k_B T^2} - \frac{\Delta(T)}{k_B T \varepsilon_k}\frac{d\Delta}{dT}\right)}{(e^{\varepsilon_k(T)/k_B T}+1)^2} = f_k(1-f_k)\left(\frac{\varepsilon_k}{k_B T^2} - \frac{\Delta(T)}{k_B T \varepsilon_k}\frac{d\Delta}{dT}\right)\]

\[ C = 2 k_B T \sum_k \frac{\varepsilon_k}{k_B T} f_k(1-f_k)\left(\frac{\varepsilon_k}{k_B T^2} - \frac{\Delta(T)}{k_B T \varepsilon_k}\frac{d\Delta}{dT}\right) \\ = 2 \sum_k \left(\frac{\varepsilon_k^2}{k_B T^2} - \frac{\Delta(T)}{k_B T}\frac{d\Delta}{dT}\right)f_k(1-f_k) \\ = \frac{2}{k_B T^2}\sum_k \left(\varepsilon_k^2 - T \Delta \frac{d\Delta}{dT}\right)f_k(1-f_k) \\ = \frac{2}{k_B T^2} \mathcal{N}(0)\int_0^\infty d\xi f(\varepsilon) [1-f(\varepsilon)]\left(\varepsilon^2 - T \Delta \frac{d\Delta}{dT}\right)\]
