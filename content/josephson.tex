Title: Josephson effect through a quantum dot Date: 2019-11-11 10:20
Category: Physics Tags: Quantum Statistics, Superconductivity Slug:
Superconductivity Author: Rémy Torro Summary: Computations related to
the Josephson effect through a quantum dot.

\subsection{Josephson effect through a quantum
dot}\label{josephson-effect-through-a-quantum-dot}

\textbf{1) Show that \(H_D\), \(H_j\), \(H_T\) without spinor notation
correspond to the same, in spinor notation.}

The total Hamiltonian is written as:

\[ H = H_D + \sum_{j = L, R} H_j + H_T\]

\textbf{\(H_D\) term:}

\[ H_D = \sum_\sigma \varepsilon d^\dagger_\sigma d_\sigma \]

We want to prove that this is equivalent to
\(H_D = \varepsilon d^\dagger \sigma_z d\), where:

\[ d = \begin{pmatrix}d_{\uparrow} & d_{\downarrow}^\dagger \end{pmatrix}^T, \qquad \sigma_z = \begin{pmatrix}1 & 0 \\
0 & -1 \end{pmatrix} \]

Let's inject this spinor into the expected spinor form of the
Hamiltonian:

\[ H_D = \varepsilon \begin{pmatrix}d_{\uparrow}^\dagger & d_{\downarrow}\end{pmatrix} \begin{pmatrix}1 & 0\\ 0 & -1\end{pmatrix}\begin{pmatrix}d_{\uparrow} \\ d_{\downarrow}^\dagger \end{pmatrix} = \varepsilon (d_{\uparrow}^\dagger d_{\uparrow} - d_{\downarrow}d_{\downarrow}^\dagger) = \varepsilon (d_{\uparrow}^\dagger d_{\uparrow} + d_{\downarrow}^\dagger d_{\downarrow} -1) = \sum_\sigma \varepsilon d^\dagger_\sigma d_\sigma - \varepsilon\]

where we used the property
\(\{d_\sigma, d^\dagger_{\sigma'}\} = \delta_{\sigma \sigma'}\).
Eventually, we can drop out the constant \(\varepsilon\) term that can
be thought of as a potential's zero energy that is totally adjustable,
as only its derivative has physical significance ({[}@coleman\_2015{]}).

\textbf{\(H_j\) term:}

The lead Hamiltonian is expressed as:

\[ H_j = \sum_{k,\sigma} \xi_k \psi_{jk\sigma}^\dagger \psi_{jk\sigma} + \Delta \sum_k (\psi_{jk\uparrow}^\dagger \psi_{jk\downarrow}^\dagger + \psi_{jk\downarrow}\psi_{jk\uparrow})\]

We want to check that this expression can be written in spinor notation
as:

\[ H_j = \sum_k \psi_{jk}^\dagger (\xi_k \sigma_z + \Delta \sigma_x)\psi_{jk}\]

where \[\psi_{jk} = \begin{pmatrix}
\psi_{jk\uparrow} & \psi_{jk\downarrow}^\dagger\end{pmatrix}^T, \qquad \sigma_x = \begin{pmatrix}0 & 1 \\
1 & 0\end{pmatrix}\]

We can substitute the full expression for the spinor in the expression
for the Hamiltonian in spinor notations:

\[ H_j = \sum_k \begin{pmatrix}\psi_{jk\uparrow}^\dagger & \psi_{jk\downarrow}\end{pmatrix}\begin{pmatrix}\xi_k & \Delta \\
\Delta & -\xi_k \end{pmatrix}\begin{pmatrix}\psi_{jk\uparrow}\\\psi_{jk\downarrow}^\dagger\end{pmatrix} 
= \sum_k \xi_k \psi_{jk\uparrow}^\dagger \psi_{jk\uparrow} + \Delta \psi_{jk\downarrow}\psi_{jk\uparrow} + \Delta \psi_{jk\uparrow}^\dagger \psi_{jk\downarrow}^\dagger - \xi_k \psi_{jk\downarrow}\psi_{jk\downarrow}^\dagger 
= \sum_k \xi_k \left[ \psi_{jk\uparrow}^\dagger \psi_{jk\uparrow} - \psi_{jk\downarrow}\psi_{jk\downarrow}^\dagger \right] + \Delta \left[\psi_{jk\downarrow}\psi_{jk\uparrow} + \psi_{jk\uparrow}^\dagger \psi_{jk\downarrow}^\dagger \right]\]

Since \$ \{\psi\emph{\{k\sigma\} , \psi}\{k'\sigma'\}\^{}\dagger \} =
\delta\_\{\sigma\sigma`\}\delta(k-k')\$, we can write:

\[ H_j = \sum_k \xi_k \left[\psi_{jk\uparrow}^\dagger \psi_{jk\uparrow} +  \psi_{jk\downarrow}^\dagger\psi_{jk\downarrow} - 1 \right] + \Delta \left[\psi_{jk\downarrow}\psi_{jk\uparrow} + \psi_{jk\uparrow}^\dagger \psi_{jk\downarrow}^\dagger \right] = \sum_{k,\sigma} \xi_k \psi_{jk\sigma}^\dagger \psi_{jk\sigma} + \sum_k \Delta \left[\psi_{jk\downarrow}\psi_{jk\uparrow} + \psi_{jk\uparrow}^\dagger \psi_{jk\downarrow}^\dagger \right] - \sum_k \xi_k\]

Eventually, as above, we can drop the constant \(\sum_k \xi_k\) term,
that will not affect the Hamiltonian, in order to recover the original
expression for \(H_j\).

\textbf{\(H_T\) term:}

The tunnelling Hamiltonian can be expressed as:

\[ H_T = \sum_{jk\sigma} \left( T_{jk} d_\sigma^\dagger \psi_{jk\sigma}+T_{jk}^* \psi_{jk\sigma}^\dagger d_\sigma\right)\]

Assuming that \(T_{jk}\) depends weakly on \(k\), we want to express
\(H_T\) in Nambu spinor notation as:

\[ H_T = \sum_{jk} \left(\psi_{jk}^\dagger T_j d + d^\dagger T_j^\dagger \psi_{jk} \right)\]

where \(T_j\) is a \(2\times 2\) matrix defined as follows:

\[T_j =\left\{
                \begin{array}{ll}
                  t_L \sigma_z e^{i \sigma_z \phi/4} \quad j = L \\
                  t_R \sigma_z e^{-i \sigma_z \phi / 4} \quad j = R \\
                \end{array}
              \right. 
\]

\(t_{L / R}\) is assumed to be a real coefficient, \(\phi\) is the phase
difference between the superconductors. Let's replace the \(L\) and
\(R\) indices with a \(\pm\), where the upper sign corresponds to
\(j=L\) and the lower sign to \(j=R\). We can immediately rework \(T_j\)
using the expression for the exponential of a Pauli matrix:
\(\exp(i \phi \sigma_i) = \mathbb{I}\cos{\phi}+i \sigma_i \sin{\phi}\).

\[ T_\pm = t_\pm \begin{pmatrix}1 & 0 \\ 0 & -1\end{pmatrix}\left[\mathbb{I}\cos{\pm \frac{\phi}{4}}+i\begin{pmatrix}1 & 0 \\ 0 & -1\end{pmatrix}\sin{\pm \frac{\phi}{4}} \right] = t_\pm \begin{pmatrix}1 & 0 \\ 0 & -1\end{pmatrix} \begin{pmatrix}\cos{\frac{\phi}{4}}\pm i\sin{\frac{\phi}{4}} & 0 \\
0 & \cos{\frac{\phi}{4}}\mp i \sin{\frac{\phi}{4}}\end{pmatrix} = t_\pm \begin{pmatrix}1 & 0 \\ 0 & -1\end{pmatrix} \begin{pmatrix} e^{\pm i \frac{\phi}{4}} & 0 \\
0 & e^{\mp i \frac{\phi}{4}}\end{pmatrix}\]

Eventually, we have the following expression for \(T_j = T_\pm\):

\[ T_j = t_j \begin{pmatrix} e^{\pm i \frac{\phi}{4}} & 0 \\
0 & -e^{\mp i \frac{\phi}{4}}\end{pmatrix}\]

We can plug this expression into the Hamiltonian in spinor notation
above:

\[ H_T = \sum_{j,k} t_j \left[\begin{pmatrix}\psi_{jk\uparrow}^\dagger & \psi_{jk\downarrow}\end{pmatrix}\begin{pmatrix}e^{\pm i \frac{\phi}{4}} & 0 \\
0 & -e^{\mp i \frac{\phi}{4}}\end{pmatrix} \begin{pmatrix}d_\uparrow \\ d_{\downarrow}^\dagger \end{pmatrix} + \begin{pmatrix}d_\uparrow^\dagger & d_\downarrow \end{pmatrix}\begin{pmatrix}e^{\mp i \frac{\phi}{4}} & 0 \\
0 & -e^{\pm i \frac{\phi}{4}}\end{pmatrix}\begin{pmatrix}\psi_{jk\uparrow}\\ \psi_{jk\downarrow}^\dagger \end{pmatrix} \right] = \sum_{j,k} \left[t_j e^{\pm i\frac{\phi}{4}}\left(\psi_{jk\uparrow}^\dagger d_\uparrow - d_\downarrow \psi_{jk\downarrow}^\dagger \right) + t_j e^{\mp i \frac{\phi}{4}} \left(-\psi_{jk\downarrow}d_\downarrow^\dagger + d_\uparrow^\dagger \psi_{jk\uparrow}  \right)\right]\]

Assuming that \(d_\alpha \sim \psi_{\alpha}^\dagger\), we can write:
\(\{d_\downarrow,\psi_{jk\downarrow}^\dagger\} = 0\), which leads to
\(d_\downarrow \psi_{jk\downarrow}^\dagger = - \psi_{jk\downarrow}^\dagger d_\downarrow\).
Furthermore, we will call \(T_{jk}^* = t_j e^{\pm i \frac{\phi}{4}}\),
such that \(T_{jk} = t_j e^{\mp i \frac{\phi}{4}}\).

Eventually, we can write:

\[ H_T = \sum_{jk\sigma} \left[T_{jk}^* \left(\psi_{jk\sigma}^\dagger d_\sigma \right) + T_{jk} \left(d_\sigma^\dagger \psi_{jk\sigma} \right)\right]\]

We recover the original expression for the tunnelling Hamiltonian.

\textbf{2) Integrating out the lead degrees of freedom, derive (using
Gaussian integration), the effective action of the dot.}

We are now interested in representing the partition function
\(Z =\textrm{Tr} \  e^{-\beta H}\) with Grassmann variables. We expect:

\[ Z = \int \mathcal{D}(\bar{\psi}\psi)\mathcal{D}(\bar{d}d)e^{-S}\]

where \(S\) is the Euclidean action and can be expanded as:

\[S = S_D + S_j + S_T = \int_0^\beta d\tau \bar{d}(\partial_\tau +\varepsilon \sigma_z) d + \int_0^\beta d\tau \left[ \sum_{jk} \bar{\psi}_{jk}(\partial_\tau + \xi_k \sigma_z +\Delta \sigma_x)\psi_{jk} \right] + \int_0^\beta d\tau \left[ \sum_{jk} \bar{\psi}_{jk} T_j d + \bar{d} T_j^\dagger \psi_{jk}\right]\]

The action has the following structure:

\[ S = \int d\tau \sum_{jk} \bar{\psi}_{jk}(\tau) M_k \psi_{jk}(\tau) + \bar{d}(\tau) N d(\tau) + \sum_{jk} \left(\bar{\psi}_{jk}(\tau) T_{jk} d(\tau) + \bar{d}(\tau) T_{jk}^\dagger \psi_{jk}(\tau) \right)\]

where \(M_k = (\partial_\tau + \xi_k \sigma_z + \Delta \sigma_x)\) and
\(N = (\partial_\tau + \varepsilon \sigma_z)\). This is a fermionic
Gaussian integral for the partition function \(Z\). We want to find the
\(\psi\) and \(\bar{\psi}\) which minimize the action \(S\), in order to
shift the Grassman variables
\(\psi(\tau) \rightarrow \psi(\tau) + \psi^c(\tau)\) and
\(\bar{\psi}(\tau) \rightarrow \bar{\psi}(\tau) + \bar{\psi}^c(\tau)\).

\[ \left. \frac{\partial S}{\partial \bar{\psi}_{jk}(\tau)}\right\vert_{\psi_{jk} = \psi_{jk}^c} = M_k(\tau) \psi_{jk}^c(\tau)+T_{jk}d(\tau) = 0 \]

\[ \left. \frac{\partial S}{\partial \psi_{jk}(\tau)}\right\vert_{\bar{\psi}_{jk} = \bar{\psi}_{jk}^c} = \bar{\psi}^c_{jk}(\tau) M_k + \bar{d}(\tau) T_{jk}^\dagger = 0 \]

\(M_k\) is an operator whose inverse in the Green's function of the
lead:

\[ M_k(\tau) G_k (\tau,\tau') = \delta(\tau - \tau')\]

The maximum is reached when

\[ \psi_{jk}^c(\tau) = - \int d\tau' G_k (\tau - \tau') T_{jk} d(\tau') \]

\[ \bar{\psi}_{jk}^c (\tau) = - \int d\tau' \bar{d}(\tau') T_{jk}^\dagger G_k^*(\tau, \tau')\]

We can operate the translation in the quadratic terms:

\[ \int d\tau \sum_{jk} (\bar{\psi}_{jk}+\bar{\psi}_{jk}^c) M_k (\psi_{jk}+\psi_{jk}^c) = \int d\tau \sum_{jk} \left[ \bar{\psi}_{jk} M_k \psi_{jk} + \bar{\psi}_{jk} M_k \psi_{jk}^c + \bar{\psi}_{jk}^c M_k \psi_{jk} + \bar{\psi}_{jk}^c M_k \psi_{jk}^c \right]= \int d\tau \sum_{jk} \left[\bar{\psi}_{jk} M_k \psi_{jk} - \int d\tau' \bar{\psi}_{jk}(\tau) \underbrace{M_k(\tau) G_k(\tau-\tau')}_{\delta(\tau - \tau')} T_{j,k} d(\tau') - \int d\tau' \bar{d}(\tau') T_{jk}^\dagger \underbrace{G_k^*(\tau-\tau') M_k(\tau)}_{\delta(\tau - \tau')}\psi_{jk}(\tau) + \int d\tau' \int d\tau'' \bar{d}(\tau') T_{jk}^\dagger G^*_k(\tau - \tau') \underbrace{M_k(\tau) G_k(\tau-\tau'')}_{\delta(\tau'' - \tau)}T_{jk}d(\tau'')  \right] = \int d\tau \sum_{jk} \left[\bar{\psi}_{jk} M_k \psi_{jk} - \bar{\psi}_{jk}(\tau) T_{jk} d(\tau) - \bar{d}(\tau) T_{jk}^\dagger \psi_{jk}(\tau) + \int d\tau' \bar{d}(\tau') T_{jk}^\dagger G^*_k(\tau - \tau') T_{jk} d(\tau) \right]\]

We can do the same translation in the tunnelling term:

\[\begin{align}\int d\tau \sum_{jk} \left[(\bar{\psi}_{jk}(\tau) + \bar{\psi}_{jk}^c)T_{jk}d(\tau) + d^\dagger(\tau)(\psi_{jk}(\tau) + \psi_{jk}^c(\tau)) \right] \\ = \int d\tau \sum_{jk} \left[\bar{\psi}_{jk}(\tau)T_{jk}d(\tau) + \bar{\psi}_{jk}^c(\tau) T_{jk} d(\tau) + d^\dagger(\tau)T_{jk}^\dagger \psi_{jk}(\tau) + d^\dagger(\tau)T_{jk}^\dagger \psi_{jk}^c(\tau)  \right] \\ = \int d\tau \sum_{jk} \left[\bar{\psi}_{jk}(\tau)T_{jk}d(\tau) - \int d\tau' \bar{d}(\tau')T_{jk}^\dagger G_k^*(\tau-\tau')T_{jk}d(\tau) + \bar{d}(\tau)T_{jk}^\dagger \psi_{jk}(\tau) - \int d\tau' \bar{d}(\tau) T_{jk}^\dagger G_k(\tau - \tau')T_{jk}d(\tau')  \right]\end{align}\]

\[\begin{align}S = \int d\tau \left\{\sum_{jk} \left[\bar{\psi}_{jk} M_k \psi_{jk} - \bar{\psi}_{jk}(\tau) T_{jk} d(\tau) - \bar{d}(\tau) T_{jk}^\dagger \psi_{jk}(\tau) \\ + \int d\tau' \bar{d}(\tau') T_{jk}^\dagger G^*_k(\tau - \tau') T_{jk} d(\tau) + \bar{\psi}_{jk}(\tau)T_{jk}d(\tau) \\ - \int d\tau' \bar{d}(\tau')T_{jk}^\dagger G_k^*(\tau-\tau')T_{jk}d(\tau) + \bar{d}(\tau)T_{jk}^\dagger \psi_{jk}(\tau) \\ - \int d\tau' \bar{d}(\tau) T_{jk}^\dagger G_k(\tau - \tau')T_{jk}d(\tau')  \right] + \bar{d}(\tau) N d(\tau)   \right\}\end{align} \]

\[ S = \int d\tau \left\{\sum_{jk} \left[\bar{\psi}_{jk} M_k \psi_{jk} - \int d\tau' \bar{d}(\tau) T_{jk}^\dagger G_k(\tau - \tau')T_{jk}d(\tau')  \right] + \bar{d}(\tau) N d(\tau)   \right\}\]

Now that the crossed terms \((\psi_{jk},d)\) have been cancelled, we
would like to perform the integration of \(Z\) on the superconducting
degrees of freedon \((\bar{\psi},\psi)\).

\[Z_S = \int \mathcal{D}(\bar{\psi}\psi)\mathcal{D}(\bar{d}d)e^{-\int d\tau \sum_{jk} \bar{\psi}_{jk}M_k\psi_{jk}}\]

\(M_k\) is some Hermitian differential operator with orthonormal
eigenstates \(\chi_{kn}\) and eigenvalues \(\lambda_{kn}\):

\[ M_k \chi_{kn} = \lambda_{kn}\chi_{kn}\]

We can expand \(\psi\) and \(\bar{\psi}\) in terms of these eigenstates:

\[ \psi(\tau) = \sum_n c_n \chi_{n}(\tau) \qquad \bar{\psi}(\tau) = \sum_n \bar{c}_n \chi^\dagger_n(\tau)\]

\[
Z_S = \int \prod_{j,k}\prod_n d\bar{c}_n dc_n e^{-\sum_{m,n} \lambda_n \bar{c}_m c_n \int d\tau \chi_m^\dagger(\tau)\chi_n(\tau)}
\\ = \int \prod_{j,k,n} d\bar{c}_n dc_n e^{-\sum_n \lambda_n \bar{c}_n c_n}
\\ = \int \prod_{j,k,n} d\bar{c}_n dc_n \prod_n (1 - \lambda_n \bar{c}_n c_n)
\\ = \prod_{j,k,n} \lambda_n = \det M \]

We are left with:

\[Z = Z_S \int \mathcal{D}(\bar{d}d)e^{-\int d\tau \bar{d}(\tau)N d(\tau) + \int d\tau \int d\tau' \bar{d}(\tau)\sum_{jk}T_{jk}^\dagger G_k(\tau - \tau') T_{jk} d(\tau')}\]

We can introduce
\(\sum (\tau - \tau') = \sum_{jk} T_{jk}^\dagger G_k(\tau - \tau') T_{jk}\),
the self energy due to the leads.

\[Z = Z_S \int \mathcal{D}(\bar{d}d)e^{-\int d\tau \bar{d}(\tau)(\partial_\tau - \varepsilon \sigma_z) d(\tau) + \int d\tau \int d\tau' \bar{d}(\tau)\sum(\tau - \tau') d(\tau')}\]

\textbf{3) Using Matsubara representation, derive the self energy of the
dot}

The Green's function of the leads satisfies the differential equation:

\[M_k G_k(\tau,\tau') = \delta(\tau-\tau')\]

\[(\partial_\tau + \xi_k \sigma_z + \Delta \sigma_x) G_k(\tau,\tau') = \delta(\tau - \tau')\]

it is noted as

\[ G_L(\tau) = (\partial_\tau + \xi_k \sigma_z + \Delta \sigma_x)^{-1} \delta(\tau)\]

We now have a quadratic effective action in terms of the dot degrees of
freedom. We will use the Matsubara representation (Fourier series for
the dot fields, with antiperiodic boundary conditions).

\[ \delta(\tau) = \frac{1}{\beta} \sum_{\omega_n} e^{-i\omega_n \tau} \qquad \omega_n = \frac{2\pi}{\beta}\left(n+\frac{1}{2}\right) \]

The Matsubara representation of the Green's function is then obtained
as:

\[ G(\tau) = \frac{1}{\beta} \sum_{\omega_n} e^{-i\omega_n \tau} G_{\omega_n}\]

\[ G_{\omega_n} = \sum_k \left(-i\omega_n + \xi_k \sigma_z + \Delta \sigma_x  \right)^{-1} \\ = \sum_k \underbrace{\begin{pmatrix}-i\omega_n + \xi_k & \Delta \\ \Delta & -i\omega_n -\xi_k \end{pmatrix}^{-1}}_{M}\]

We can compute the inverse of this matrix, assuming that it is
invertible \emph{i.e.} \(\det{M}\neq 0\).

\[ M^{-1} = \frac{1}{\det{M}} C^T\]

where \(C^T\) is the transpose of the cofactor matrix of \(M\).

\[\det{M} = (-i\omega_n + \xi_k)(-i\omega_n - \xi_k)-\Delta^2 = -(\omega_n^2 + \xi_k^2 + \Delta^2)\]

\[ M^{-1} = \frac{-1}{\omega_n^2 + \xi_k^2 + \Delta^2} \begin{pmatrix}-i\omega_n - \xi_k & -\Delta \\ -\Delta & -i\omega_n + \xi_k \end{pmatrix} = \frac{1}{\omega_n^2 + \xi_k^2 + \Delta^2} \begin{pmatrix}i\omega_n + \xi_k & \Delta \\ \Delta & i\omega_n - \xi_k \end{pmatrix} \\ = \frac{i\omega_n \sigma_0 + \xi_k \sigma_z + \Delta \sigma_x}{\omega_n^2 + \xi_k^2 + \Delta^2}\]

In the continuum limit,
\(\sum_k \rightarrow \int d\xi \nu(\xi) \sim \int d\xi \nu(0)\). We can
perform the integration separately for each term in the numerator, using
the property \(\int d\xi (\xi^2 + b^2)^{-1} = \pi/b\).

\[ i \omega_n \sigma_0 \int d\xi \nu(0) \frac{1}{(\omega_n^2 + \Delta^2)+\xi^2} = i\omega_n \sigma_0 \nu(0) \frac{\pi}{\sqrt{\omega_n^2 + \Delta^2}} \]

\[ \sigma_z \int_{-\infty}^{\infty} d\xi \nu(0) \underbrace{\frac{\xi}{(\omega_n^2 + \Delta^2)+\xi^2}}_{\textrm{odd function}} = 0\]

\[ \Delta \sigma_x \int d\xi \nu(0) \frac{1}{(\omega_n^2 + \Delta^2)+\xi^2} = \Delta \sigma_x \nu(0) \frac{\pi}{\sqrt{\omega_n^2 + \Delta^2}} \]

So we can write:

\[ G_{\omega_n} = \frac{\pi \nu(0)}{\sqrt{\omega_n^2 + \Delta^2}}(i\omega_n \sigma_0 + \Delta \sigma_x)\]

We can now derive the self-energy, using \(G_{\omega_n}\) and the
previously defined \(T_j\) matrix:

\[ \sum(\omega_n) = \frac{\pi \nu(0)}{\sqrt{\omega_n^2 + \Delta^2}}\sum_{j=L,R} T_j^\dagger G_{\omega_n} T_j
\\ = \frac{\pi \nu(0)}{\sqrt{\omega_n^2 + \Delta^2}} \sum_{j=L,R} t_j^2 \begin{pmatrix}e^{\mp i \phi/4} & 0 \\ 0 & -e^{\pm i \phi/4} \end{pmatrix}\begin{pmatrix}i\omega_n & \Delta \\ \Delta & i\omega_n \end{pmatrix}\begin{pmatrix}e^{\pm i \phi/4} & 0 \\ 0 & -e^{\mp i \phi/4}\end{pmatrix} \\ = \frac{\pi \nu(0)}{\sqrt{\omega_n^2 + \Delta^2}} \sum_{j=L,R} t_j^2 \begin{pmatrix}i\omega_n & -\Delta e^{\mp i \phi/2} \\ -\Delta e^{\pm i \phi/2} & i\omega_n \end{pmatrix} \\ = \frac{\pi \nu(0)}{\sqrt{\omega_n^2 + \Delta^2}} \left\{ t_L^2 \underbrace{\begin{pmatrix}i\omega_n & -\Delta[\cos(\phi/2) - i \sin(\phi/2)] \\ -\Delta[\cos(\phi/2)+i\sin(\phi/2)] & i\omega_n \end{pmatrix}}_{i\omega_n \sigma_0 - \Delta \cos(\phi/2)\sigma_x - \Delta \sin(\phi/2)\sigma_y} + t_R^2 \underbrace{\begin{pmatrix}i\omega_n & -\Delta[\cos(\phi/2) + i \sin(\phi/2)] \\ -\Delta[\cos(\phi/2)-i\sin(\phi/2)] & i\omega_n \end{pmatrix}}_{i\omega_n \sigma_0 -\Delta \cos(\phi/2)\sigma_x + \Delta \sin(\phi/2) \sigma_y}\right\} \\ = \frac{\pi \nu(0)}{\sqrt{\omega_n^2 + \Delta^2}} \left\{(t_L^2 + t_R^2)[i\omega_n\sigma_0 - \Delta \cos(\phi/2)\sigma_x] - \Delta\sin(\phi/2) \sigma_y (t_L^2 - t_R^2)\right\} \\ = \frac{\pi \nu(0)}{\sqrt{\omega_n^2 + \Delta^2}}  (t_L^2 + t_R^2)\left[i\omega_n\sigma_0 - \Delta \cos(\phi/2)\sigma_x - \frac{t_L^2 - t_R^2}{t_L^2 + t_R^2} \Delta\sin(\phi/2) \sigma_y \right] \\ = \frac{\pi \nu(0)}{\sqrt{\omega_n^2 + \Delta^2}}  (t_L^2 + t_R^2)\left[i\omega_n\sigma_0 - \Delta \cos(\phi/2)\sigma_x - \gamma \Delta\sin(\phi/2) \sigma_y \right] \]

where \(\gamma = (t_L^2 - t_R^2)/(t_L^2 + t_R^2)\). As a result, the
self energy of the dot can be expressed as:

\[ \Sigma_{\omega_n} = \frac{\pi \nu(0) (t_L^2 + t_R^2)}{\sqrt{\omega_n^2 + \Delta^2}} \left[i\omega_n\sigma_0 - \Delta \cos(\phi/2)\sigma_x - \gamma \Delta\sin(\phi/2) \sigma_y \right] \]

\[ \Sigma_{\omega_n} = \frac{\Gamma / 2}{\sqrt{\Delta^2 - (i\omega_n)^2}} \left[i\omega_n\sigma_0 - \Delta \cos(\phi/2)\sigma_x - \gamma \Delta\sin(\phi/2) \sigma_y \right] \]

where we introduced \(\Gamma = 2\pi \nu(0) (t_L^2 + t_R^2)\).

\textbf{4) Compute the fermionic determinant for a given Matsubara
frequency}

Now we integrate out the dot degrees of freedom:

\[Z = \int \mathcal{D}(\bar{d}d)e^{-S_{\textrm{eff}}} \\ = \int \mathcal{D}(\bar{d}d) e^{-\int_0^\beta d\tau \bar{d}(\tau)(\partial_\tau + \varepsilon \sigma_z)d(\tau)} e^{-\int_0^\beta d\tau \int_0^\beta d\tau' \bar{d}(\tau) \Sigma (\tau - \tau') d(\tau')} \\ = \int \mathcal{D}(\bar{d}d)e^{-\sum_{\omega_n} \bar{d}_{\omega_n} M_{\omega_n} d_{\omega_n}} \\ = \prod_{\omega_n} \det M_{\omega_n} \]

with
\(M_{\omega_n} = -i\omega_n +\varepsilon \sigma_z - \Sigma_{\omega_n}\).
Since

\[ J = -\frac{2}{\beta} \frac{\partial}{\partial \phi} \ln Z \\ = -\frac{2}{\beta} \frac{\partial}{\partial \phi} \sum_{\omega_n} \ln \det M_{\omega_n}\]

We will start by expressing \(M_{\omega_n}\) in matrix notation:

\[M_{\omega_n} = \begin{pmatrix}-i\omega_n + \varepsilon & 0 \\ 0  & -i\omega_n -\varepsilon \end{pmatrix} - \frac{\Gamma / 2}{\sqrt{\Delta^2 -(i\omega)^2}}\begin{pmatrix}i\omega_n & -\Delta \cos{\phi/2} + i \gamma \Delta \sin{\phi/2} \\ -\Delta \cos{\phi/2} - i \gamma \Delta \sin{\phi/2} & i\omega_n \end{pmatrix} = \begin{pmatrix}\frac{(-i\omega_n + \varepsilon)\sqrt{\Delta^2 -(i\omega_n)^2} - \Gamma/2 i \omega_n}{\sqrt{\Delta^2 -(i\omega_n)^2}} & \frac{-\Gamma / 2 (-\Delta \cos{\phi/2} + i \gamma \Delta \sin{\phi/2})}{\sqrt{\Delta^2 -(i\omega_n)^2}} \\ \frac{-\Gamma / 2 (-\Delta \cos{\phi/2} - i \gamma \Delta \sin{\phi/2})}{\sqrt{\Delta^2 -(i\omega_n)^2}} & \frac{(-i\omega_n - \varepsilon)\sqrt{\Delta^2 -(i\omega_n)^2} - \Gamma / 2 i\omega_n}{\sqrt{\Delta^2 -(i\omega_n)^2}} \end{pmatrix} \]

\[\det M_{\omega_n} = \underbrace{(-i\omega_n + \varepsilon)(-i\omega_n - \varepsilon)}_{(a+b)(a-b) = a^2 - b^2} - \frac{\Gamma / 2 i \omega_n \left[(-i\omega_n + \varepsilon) + (-i\omega_n - \varepsilon)\right]}{\sqrt{\Delta^2 -(i\omega_n)^2}} + \frac{(\Gamma / 2)^2 (i\omega)^2}{\Delta^2 -(i\omega)^2} - \frac{(\Gamma / 2)^2}{\Delta^2 -(i\omega_n)^2}\underbrace{(-\Delta \cos{\phi/2}+i\gamma\Delta\sin{\phi/2})(-\Delta \cos{\phi/2}-i\gamma \Delta \sin{\phi/2})}_{zz^* = |z|^2} \\ = (i\omega_n)^2 - \varepsilon^2 + \frac{(i\omega_n)^2 \Gamma}{\sqrt{\Delta^2 -(i\omega_n)^2}}+ \frac{(\Gamma / 2)^2 (i\omega)^2}{\Delta^2 -(i\omega)^2}- \frac{(\Gamma / 2)^2}{\Delta^2 -(i\omega_n)^2}\left[\Delta^2 \cos^2{\phi/2} + \gamma^2 \Delta^2 \sin^2{\phi/2} \right] \\ = - \varepsilon^2 + (i\omega_n)^2 \underbrace{\left[1 + \frac{\Gamma}{\sqrt{\Delta^2 -(i\omega_n)^2}}+ \frac{(\Gamma / 2)^2 }{\Delta^2 -(i\omega)^2}\right]}_{= \left[1 + (\Gamma/2)/(\sqrt{\Delta^2-(i\omega_n)^2})\right]^2} - \frac{(\Gamma / 2)^2}{\Delta^2 -(i\omega_n)^2}\left[\Delta^2 \cos^2{\phi/2} + \gamma^2 \Delta^2 \sin^2{\phi/2} \right] \]

\[ \det M_{\omega_n} = (i\omega_n)^2 \left[1 + \frac{\Gamma/2}{\sqrt{\Delta^2-(i\omega_n)^2}}\right]^2 - \varepsilon^2 -  \frac{(\Gamma / 2)^2 \Delta^2}{\Delta^2 -(i\omega_n)^2}\left[\cos^2{\phi/2} + \gamma^2 \sin^2{\phi/2} \right]\]

Eventually, the Josephson current can be written as:

\[ J = -\frac{2}{\beta} \frac{\partial }{\partial \phi} \sum_{\omega_n} \ln \left\{(i\omega_n)^2 \left[1 + \frac{\Gamma/2}{\sqrt{\Delta^2-(i\omega_n)^2}}\right]^2 - \varepsilon^2 -  \frac{(\Gamma / 2)^2 \Delta^2}{\Delta^2 -(i\omega_n)^2}\left[\cos^2{\phi/2} + \gamma^2 \sin^2{\phi/2} \right]\right\}\]

\textbf{5) Derive the equation for the Andreev levels, and find the
approximate solution for \(Z\)}

The Andreev levels are defined as the poles of the dot Green's function.
Since the effective action in Matsubara representation reads:

\[ S_{\textrm{eff}} = \sum_{\omega_n} \bar{d}_{\omega_n} M_{\omega_n} d_{\omega_n}\]

and \(M^{-1}_{\omega_n} = G_{\textrm{dot}}\). Finding these poles is
equivalent to finding the zeros of the determinant of \(M_{\omega_n}\)
for \(z=i\omega\):

\[ \det\left[z - \varepsilon \sigma_z + \frac{\Gamma /2}{\sqrt{\Delta^2 - z^2}}(z - \Delta \cos(\phi/2) \sigma_x - \gamma \Delta \sin(\phi/2)\sigma_y)\right] = 0\]

But we computed this determinant before:

\[z^2 \left[1 + \frac{\Gamma/2}{\sqrt{\Delta^2-z^2}}\right]^2 - \varepsilon^2 -  \frac{(\Gamma / 2)^2 \Delta^2}{\Delta^2 -z^2}\left[\cos^2{\phi/2} + \gamma^2 \sin^2{\phi/2} \right] = 0 \]

\emph{a) in the tunnel regime \(\Gamma, \varepsilon << \Delta\) (quartic
equation) (throwing away the solution which is not consisent with this
assumption (\(Z << \Delta\))}

From \(\Delta >> \Gamma\), we will assume that
\(\frac{\Gamma /2}{\sqrt{\Delta^2 - z^2}} = \frac{\Gamma/2}{\Delta\sqrt{1 - z^2/\Delta^2}} \rightarrow 0\)
quickly. The equation that we must solve is now:

\[ z^2 - \varepsilon^2 -  \frac{(\Gamma / 2)^2 \Delta^2}{\Delta^2 -z^2}\left[\cos^2{\phi/2} + \gamma^2 \sin^2{\phi/2} \right] = 0\]

\[ (\Delta^2 - z^2) z^2 - (\Delta^2 - z^2)\varepsilon^2 - \frac{\Gamma^2}{4}\Delta^2\left[\cos^2(\phi/2)+\gamma^2\sin^2(\phi/2)\right] = 0\]

\[ z^4 - (\Delta^2 + \varepsilon^2)z^2 + \Delta^2 \left(\varepsilon^2 + \frac{\Gamma^2}{4}\left[\cos^2(\phi/2)+\gamma^2\sin^2(\phi/2)\right]\right) = 0\]

Let's introduce \(X = z^2\):

\[ X^2 - (\Delta^2 + \varepsilon^2)X + \Delta^2 \left(\varepsilon^2 + \frac{\Gamma^2}{4}\left[\cos^2(\phi/2)+\gamma^2\sin^2(\phi/2)\right]\right) = 0\]

\[ D = \Delta^4+\varepsilon^4 - \Delta^2 \left[2\varepsilon^2 + \Gamma^2 \left(\cos^2(\phi/2)+\gamma^2\sin^2(\phi/2)\right)\right]\]

where \(D\) is the determinant of the polynomial, and is assumed to be
positive.

\[ X_\pm = z^2_\pm = \frac{\Delta^2 + \varepsilon^2}{2} \pm \frac{1}{2}\sqrt{\Delta^4+\varepsilon^4 - \Delta^2 \left[2\varepsilon^2 + \Gamma^2 \left(\cos^2(\phi/2)+\gamma^2\sin^2(\phi/2)\right)\right]} \\ = \frac{\Delta^2 + \varepsilon^2}{2} \pm \frac{\Delta^2}{2}\sqrt{1+\frac{\varepsilon^4}{\Delta^4} - \frac{2\varepsilon^2}{\Delta^2} - \frac{\Gamma^2}{\Delta^2} \left( \cos^2(\phi/2)+\gamma^2\sin^2(\phi/2)\right)} \sim \frac{\Delta^2 + \varepsilon^2}{2} \pm \frac{\Delta^2}{2} \left[1+\underbrace{\frac{\varepsilon^4}{2\Delta^4}}_{\rightarrow 0} - \frac{\varepsilon^2}{\Delta^2} - \frac{\Gamma^2}{2\Delta^2} \left( \cos^2(\phi/2)+\gamma^2\sin^2(\phi/2)\right) \right] \\ = \frac{\Delta^2}{2} + \frac{\varepsilon^2}{2} \pm \frac{\Delta^2}{2} \mp \frac{\varepsilon^2}{2}\mp \frac{\Gamma^2}{4}\left( \cos^2(\phi/2)+\gamma^2\sin^2(\phi/2)\right)\]

Where we used \(\sqrt{1 + x} \sim 1+x/2\) for \(x \rightarrow 0\), and
the \(\varepsilon^4/\Delta^4\) term was considered small enough to be
neglected. The \(z_+\) solution can be written as:

\[ z_+ = \pm \Delta\sqrt{1 - \frac{\Gamma^2}{4\Delta^2}\left( \cos^2(\phi/2)+\gamma^2\sin^2(\phi/2)\right)}\]

This solution \(z_+ \sim \pm \Delta\) is not consistent with the
original assumption that \(|z|<<\Delta\), so we discard it.

\[ z_-^2 = \varepsilon^2 + \frac{\Gamma^2}{4}\left( \cos^2(\phi/2)+\gamma^2\sin^2(\phi/2)\right) \\ = \varepsilon^2 + \frac{\Gamma^2}{4}(1 - \sin^2(\phi/2)+\gamma^2 \sin(\phi/2)) \\ = \varepsilon^2 + \frac{\Gamma^2}{4} \left[1 + (\gamma^2 - 1) \sin^2(\phi/2) \right] \\ = \varepsilon^2 + \frac{\Gamma^2}{4} - \frac{\Gamma^2}{4}(1-\gamma^2)\sin^2(\phi/2) \\ = \left(\varepsilon^2+\frac{\Gamma^2}{4}\right)\left(1 - \frac{1-\gamma^2}{1+\frac{4 \varepsilon^2}{\Gamma^2}}\sin^2(\phi/2)\right)  \]

As a result, we can write:

\[ z_- = \pm \sqrt{\left(\varepsilon^2 + \frac{\Gamma^2}{4}\right)}\sqrt{1 - \frac{1 - \gamma^2}{1 + \left(\frac{2\varepsilon}{\Gamma}\right)^2}\sin^2(\phi/2)}\]

\[ z_- = \pm \delta \sqrt{1-D\sin^2(\phi/2)}\]

introducing
\(\delta = \sqrt{\left(\varepsilon^2 + \frac{\Gamma^2}{4}\right)}\) and
\(D = \frac{1 - \gamma^2}{1 + \left(\frac{2\varepsilon}{\Gamma}\right)^2}\).
\(D \leq 1\) plays the role of a transmission coefficient. The Andreev
levels are far from the gap.

\emph{b) in the high transparency regime \(\Gamma >> \Delta\)}

In the high transparency regime, we can assume that
\(\Gamma >> \Delta\), such that
\(\frac{\Gamma/2}{\sqrt{\Delta^2 - z^2}}>> 1\), so we neglect the \(1\):

\[ z^2 \frac{(\Gamma/2)^2}{\Delta^2 - z^2} - \varepsilon^2 - \frac{(\Gamma/2)^2\Delta^2}{\Delta^2 - z^2}\left[\cos^2(\phi/2) + \gamma^2 \sin^2(\phi/2) \right] = 0\]

\[ z^2 (\Gamma^2/4) - \varepsilon^2 (\Delta^2 - z^2) - (\Gamma^2/4)\Delta^2\left[\cos^2(\phi/2) + \gamma^2 \sin^2(\phi/2) \right] = 0\]

\[ z^2 = \frac{\Delta^2 \left[\varepsilon^2 + (\Gamma^2/4) \left[\sin^2(\phi/2)(\gamma^2 - 1) +1\right]\right]}{(\Gamma^2/4)+\varepsilon^2} \\ = \Delta^2 \left[1 - \frac{(\Gamma^2/4)(1-\gamma^2)}{\varepsilon^2 + (\Gamma^2/4)}\sin^2(\phi/2)\right] \\ = \Delta^2 \left[1 - \frac{(1-\gamma^2)}{1 + \left(\frac{2 \varepsilon}{\Gamma}\right)^2}\sin^2(\phi/2)\right]\]

\[ z = \pm \Delta \sqrt{1 - \frac{(1-\gamma^2)}{1 + \left(\frac{2 \varepsilon}{\Gamma}\right)^2}\sin^2(\phi/2)} \\ = \pm \Delta \sqrt{1 - D\sin^2(\phi/2)}\]

This time, Andreev levels are close to the gap.
